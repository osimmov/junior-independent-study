% This is a template for your written document.
%
% To compile using latexmk on the command line, run the following: 
% latexmk -pdf main.tex

\documentclass[12pt]{article}
\usepackage{setspace}
\usepackage{graphicx} % used for includegraphics
\singlespace
\usepackage[left=1in,right=1in,top=1in,bottom=1in]{geometry}

\title{\textbf{Reflective AI Feedback for Personal Productivity}}
\author{Saidamir Osimov}

\begin{document}

\maketitle
I have used task lists and productivity systems for most of my academic life, ranging from handwritten notebooks to modern digital productivity platforms. While these tools are effective at recording tasks and reminders, they often fail to help users understand how they plan, execute, and reflect on their work over time. In my own experience, I frequently created overly ambitious task lists, leading to frustration when goals were not completed. Additionally, most productivity tools offer limited insight into long-term behavioral patterns such as task overload, inconsistent pacing, or changes in motivation throughout the week.

Recent advances in artificial intelligence, particularly large language models (LLMs), present an opportunity to move beyond static task tracking toward reflective analysis of human workflows. Prior research has shown that AI systems can support human thinking by assisting with ideation, reflection, and decision-making rather than replacing human agency \cite{10.1145/3706598.3713295}
. Similarly, work on human–AI collaboration demonstrates that AI is most effective when positioned as a cognitive partner that provides feedback and perspective \cite{10.1145/3640543.3645198}. These findings motivate the design of productivity tools that emphasize reflection and insight rather than automation alone.

This project proposes the design and implementation of a local, privacy-preserving productivity application augmented with an AI-based reflection system. The core research question guiding this work is: \textit{Does reflective feedback from an AI system influence user productivity behavior and planning habits over time?} Rather than directly executing tasks on behalf of the user, the AI will analyze task metadata and daily reflections to generate summaries, observations, and personalized suggestions regarding productivity patterns. This approach aligns with the concept of “supermind intelligence,” which describes systems that enhance collective or individual problem-solving through structured feedback and insight rather than autonomous control \cite{10.1145/3643562.3672611}.

The software will allow users to organize tasks by day of the week using a visual interface inspired by timeline-based productivity tools. Each task will include a title, description, creation time, and completion status. Users may optionally write short daily reflections describing their perceived productivity or challenges. The AI system, running locally via an open-source language model, will have access to this structured data and will periodically generate reflective feedback. Examples include identifying periods of over-planning, changes in task completion rates, or correlations between reflections and productivity outcomes.
To evaluate the effectiveness of reflective AI feedback, the project will involve a small user study. While the scope of the study will be limited, it will provide initial evidence on whether AI-driven reflection can meaningfully support personal productivity.
By focusing on reflection, transparency, and user agency, the system aims to support users in understanding their own work habits rather than delegating responsibility to automation.



\begin{figure}[h]
\begin{center}
\includegraphics[scale=0.3]{interface.png}
\caption{Interface}
\label{fig:interface}       % Give a unique label
\end{center}
\end{figure}


\newpage
\section*{Appendix}
A concise list of features / user stories in the order in which they will be built. A few examples are below to demonstrate the expected scope and level of granularity; you will have more features than this.
\begin{itemize}
	\item Default picture display on web application.
	\item On a button-click, user can separate the image into foreground and background.
	\item User can select a picture from their desktop.
	\item Selected picture displays on the web application.
\end{itemize}


\bibliographystyle{acm}
\bibliography{bibliography.bib}

\end{document}
